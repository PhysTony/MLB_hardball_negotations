% Options for packages loaded elsewhere
\PassOptionsToPackage{unicode}{hyperref}
\PassOptionsToPackage{hyphens}{url}
%
\documentclass[
]{article}
\usepackage{amsmath,amssymb}
\usepackage{lmodern}
\usepackage{ifxetex,ifluatex}
\ifnum 0\ifxetex 1\fi\ifluatex 1\fi=0 % if pdftex
  \usepackage[T1]{fontenc}
  \usepackage[utf8]{inputenc}
  \usepackage{textcomp} % provide euro and other symbols
\else % if luatex or xetex
  \usepackage{unicode-math}
  \defaultfontfeatures{Scale=MatchLowercase}
  \defaultfontfeatures[\rmfamily]{Ligatures=TeX,Scale=1}
\fi
% Use upquote if available, for straight quotes in verbatim environments
\IfFileExists{upquote.sty}{\usepackage{upquote}}{}
\IfFileExists{microtype.sty}{% use microtype if available
  \usepackage[]{microtype}
  \UseMicrotypeSet[protrusion]{basicmath} % disable protrusion for tt fonts
}{}
\makeatletter
\@ifundefined{KOMAClassName}{% if non-KOMA class
  \IfFileExists{parskip.sty}{%
    \usepackage{parskip}
  }{% else
    \setlength{\parindent}{0pt}
    \setlength{\parskip}{6pt plus 2pt minus 1pt}}
}{% if KOMA class
  \KOMAoptions{parskip=half}}
\makeatother
\usepackage{xcolor}
\IfFileExists{xurl.sty}{\usepackage{xurl}}{} % add URL line breaks if available
\IfFileExists{bookmark.sty}{\usepackage{bookmark}}{\usepackage{hyperref}}
\hypersetup{
  pdftitle={Variables de control - Agentes libres},
  pdfauthor={Antonio Huerta Montellano},
  hidelinks,
  pdfcreator={LaTeX via pandoc}}
\urlstyle{same} % disable monospaced font for URLs
\usepackage[margin=1in]{geometry}
\usepackage{color}
\usepackage{fancyvrb}
\newcommand{\VerbBar}{|}
\newcommand{\VERB}{\Verb[commandchars=\\\{\}]}
\DefineVerbatimEnvironment{Highlighting}{Verbatim}{commandchars=\\\{\}}
% Add ',fontsize=\small' for more characters per line
\usepackage{framed}
\definecolor{shadecolor}{RGB}{248,248,248}
\newenvironment{Shaded}{\begin{snugshade}}{\end{snugshade}}
\newcommand{\AlertTok}[1]{\textcolor[rgb]{0.94,0.16,0.16}{#1}}
\newcommand{\AnnotationTok}[1]{\textcolor[rgb]{0.56,0.35,0.01}{\textbf{\textit{#1}}}}
\newcommand{\AttributeTok}[1]{\textcolor[rgb]{0.77,0.63,0.00}{#1}}
\newcommand{\BaseNTok}[1]{\textcolor[rgb]{0.00,0.00,0.81}{#1}}
\newcommand{\BuiltInTok}[1]{#1}
\newcommand{\CharTok}[1]{\textcolor[rgb]{0.31,0.60,0.02}{#1}}
\newcommand{\CommentTok}[1]{\textcolor[rgb]{0.56,0.35,0.01}{\textit{#1}}}
\newcommand{\CommentVarTok}[1]{\textcolor[rgb]{0.56,0.35,0.01}{\textbf{\textit{#1}}}}
\newcommand{\ConstantTok}[1]{\textcolor[rgb]{0.00,0.00,0.00}{#1}}
\newcommand{\ControlFlowTok}[1]{\textcolor[rgb]{0.13,0.29,0.53}{\textbf{#1}}}
\newcommand{\DataTypeTok}[1]{\textcolor[rgb]{0.13,0.29,0.53}{#1}}
\newcommand{\DecValTok}[1]{\textcolor[rgb]{0.00,0.00,0.81}{#1}}
\newcommand{\DocumentationTok}[1]{\textcolor[rgb]{0.56,0.35,0.01}{\textbf{\textit{#1}}}}
\newcommand{\ErrorTok}[1]{\textcolor[rgb]{0.64,0.00,0.00}{\textbf{#1}}}
\newcommand{\ExtensionTok}[1]{#1}
\newcommand{\FloatTok}[1]{\textcolor[rgb]{0.00,0.00,0.81}{#1}}
\newcommand{\FunctionTok}[1]{\textcolor[rgb]{0.00,0.00,0.00}{#1}}
\newcommand{\ImportTok}[1]{#1}
\newcommand{\InformationTok}[1]{\textcolor[rgb]{0.56,0.35,0.01}{\textbf{\textit{#1}}}}
\newcommand{\KeywordTok}[1]{\textcolor[rgb]{0.13,0.29,0.53}{\textbf{#1}}}
\newcommand{\NormalTok}[1]{#1}
\newcommand{\OperatorTok}[1]{\textcolor[rgb]{0.81,0.36,0.00}{\textbf{#1}}}
\newcommand{\OtherTok}[1]{\textcolor[rgb]{0.56,0.35,0.01}{#1}}
\newcommand{\PreprocessorTok}[1]{\textcolor[rgb]{0.56,0.35,0.01}{\textit{#1}}}
\newcommand{\RegionMarkerTok}[1]{#1}
\newcommand{\SpecialCharTok}[1]{\textcolor[rgb]{0.00,0.00,0.00}{#1}}
\newcommand{\SpecialStringTok}[1]{\textcolor[rgb]{0.31,0.60,0.02}{#1}}
\newcommand{\StringTok}[1]{\textcolor[rgb]{0.31,0.60,0.02}{#1}}
\newcommand{\VariableTok}[1]{\textcolor[rgb]{0.00,0.00,0.00}{#1}}
\newcommand{\VerbatimStringTok}[1]{\textcolor[rgb]{0.31,0.60,0.02}{#1}}
\newcommand{\WarningTok}[1]{\textcolor[rgb]{0.56,0.35,0.01}{\textbf{\textit{#1}}}}
\usepackage{graphicx}
\makeatletter
\def\maxwidth{\ifdim\Gin@nat@width>\linewidth\linewidth\else\Gin@nat@width\fi}
\def\maxheight{\ifdim\Gin@nat@height>\textheight\textheight\else\Gin@nat@height\fi}
\makeatother
% Scale images if necessary, so that they will not overflow the page
% margins by default, and it is still possible to overwrite the defaults
% using explicit options in \includegraphics[width, height, ...]{}
\setkeys{Gin}{width=\maxwidth,height=\maxheight,keepaspectratio}
% Set default figure placement to htbp
\makeatletter
\def\fps@figure{htbp}
\makeatother
\setlength{\emergencystretch}{3em} % prevent overfull lines
\providecommand{\tightlist}{%
  \setlength{\itemsep}{0pt}\setlength{\parskip}{0pt}}
\setcounter{secnumdepth}{-\maxdimen} % remove section numbering
\ifluatex
  \usepackage{selnolig}  % disable illegal ligatures
\fi

\title{Variables de control - Agentes libres}
\author{Antonio Huerta Montellano}
\date{15 de junio del 2022}

\begin{document}
\maketitle

Importando las librerías:

Veamos cuál es el directorio de trabajo

\begin{Shaded}
\begin{Highlighting}[]
\FunctionTok{getwd}\NormalTok{()}
\end{Highlighting}
\end{Shaded}

\begin{verbatim}
## [1] "/home/usuario/Documentos/Github/Proyectos/MLB_HN/Models/Instrumental_variables/Free_agent"
\end{verbatim}

Cambiemos el directorio de trabajo y carguemos las bases de datos para
el modelo lineal en el mismo chunk:

\begin{Shaded}
\begin{Highlighting}[]
\FunctionTok{setwd}\NormalTok{(}\StringTok{"\textasciitilde{}/Documentos/Github/Proyectos/MLB\_HN/"}\NormalTok{)}
\NormalTok{free\_agents }\OtherTok{\textless{}{-}} \FunctionTok{read.csv}\NormalTok{(}\StringTok{\textquotesingle{}Data/New\_Data/Models/Article/article\_iv\_linear\_regression\_fa.csv\textquotesingle{}}\NormalTok{)}
\end{Highlighting}
\end{Shaded}

Observemos el contenido de las bases de datos de los agentes libres:

\begin{Shaded}
\begin{Highlighting}[]
\FunctionTok{head}\NormalTok{(free\_agents)}
\end{Highlighting}
\end{Shaded}

\begin{verbatim}
##           Jugador          Y          X Posicion Equipo Equipos_estado
## 1    A.J. Burnett  -471.0402  0.6333714       SP    PHI              2
## 2 A.J. Pierzynski   199.1753 -0.5233840        C    ATL              1
## 3 A.J. Pierzynski  1218.5682 -0.6992366        C    STL              2
## 4 A.J. Pierzynski -2515.7319  0.7312696        C    TEX              2
## 5    Aaron Harang   509.2721 -0.6291943       SP    LAD              5
## 6    Adam LaRoche   293.4260  0.6825397       DH    WSH              1
\end{verbatim}

\section{Dummificación}

Creemos las variables dummy correspondientes a las variables categóricas
de la posición del jugador y al estad

\begin{Shaded}
\begin{Highlighting}[]
\NormalTok{dummy }\OtherTok{\textless{}{-}} \FunctionTok{dummy\_cols}\NormalTok{(free\_agents, }\AttributeTok{select\_columns =} \FunctionTok{c}\NormalTok{(}\StringTok{\textquotesingle{}Posicion\textquotesingle{}}\NormalTok{, }\StringTok{\textquotesingle{}Equipo\textquotesingle{}}\NormalTok{),}
                    \AttributeTok{remove\_selected\_columns =} \ConstantTok{TRUE}\NormalTok{)}
\end{Highlighting}
\end{Shaded}

Notemos que ahora hay muchas columnas debido a la dummificación y la
gran cantidad de categorías tanto para la posición que puede ocupar un
jugador como para el equipo al que pertencen (29):

\begin{Shaded}
\begin{Highlighting}[]
\FunctionTok{head}\NormalTok{(dummy)}
\end{Highlighting}
\end{Shaded}

\begin{verbatim}
##           Jugador          Y          X Equipos_estado Posicion_1B Posicion_2B
## 1    A.J. Burnett  -471.0402  0.6333714              2           0           0
## 2 A.J. Pierzynski   199.1753 -0.5233840              1           0           0
## 3 A.J. Pierzynski  1218.5682 -0.6992366              2           0           0
## 4 A.J. Pierzynski -2515.7319  0.7312696              2           0           0
## 5    Aaron Harang   509.2721 -0.6291943              5           0           0
## 6    Adam LaRoche   293.4260  0.6825397              1           0           0
##   Posicion_3B Posicion_C Posicion_CF Posicion_DH Posicion_LF Posicion_RF
## 1           0          0           0           0           0           0
## 2           0          1           0           0           0           0
## 3           0          1           0           0           0           0
## 4           0          1           0           0           0           0
## 5           0          0           0           0           0           0
## 6           0          0           0           1           0           0
##   Posicion_RP Posicion_RP/CL Posicion_SP Posicion_SS Equipo_ARI Equipo_ATL
## 1           0              0           1           0          0          0
## 2           0              0           0           0          0          1
## 3           0              0           0           0          0          0
## 4           0              0           0           0          0          0
## 5           0              0           1           0          0          0
## 6           0              0           0           0          0          0
##   Equipo_BAL Equipo_BOS Equipo_CHC Equipo_CHW Equipo_CIN Equipo_COL Equipo_DET
## 1          0          0          0          0          0          0          0
## 2          0          0          0          0          0          0          0
## 3          0          0          0          0          0          0          0
## 4          0          0          0          0          0          0          0
## 5          0          0          0          0          0          0          0
## 6          0          0          0          0          0          0          0
##   Equipo_HOU Equipo_KC Equipo_LAA Equipo_LAD Equipo_MIA Equipo_MIL Equipo_MIN
## 1          0         0          0          0          0          0          0
## 2          0         0          0          0          0          0          0
## 3          0         0          0          0          0          0          0
## 4          0         0          0          0          0          0          0
## 5          0         0          0          1          0          0          0
## 6          0         0          0          0          0          0          0
##   Equipo_NYM Equipo_NYY Equipo_OAK Equipo_PHI Equipo_PIT Equipo_SD Equipo_SEA
## 1          0          0          0          1          0         0          0
## 2          0          0          0          0          0         0          0
## 3          0          0          0          0          0         0          0
## 4          0          0          0          0          0         0          0
## 5          0          0          0          0          0         0          0
## 6          0          0          0          0          0         0          0
##   Equipo_SF Equipo_STL Equipo_TB Equipo_TEX Equipo_TOR Equipo_WSH
## 1         0          0         0          0          0          0
## 2         0          0         0          0          0          0
## 3         0          1         0          0          0          0
## 4         0          0         0          1          0          0
## 5         0          0         0          0          0          0
## 6         0          0         0          0          0          1
\end{verbatim}

\section{Creación del modelo}

Para ello y evitar errores, lo haremos con los nombres explícitos,
obtengamos los nombres de las columnas

\begin{Shaded}
\begin{Highlighting}[]
\FunctionTok{colnames}\NormalTok{(dummy)}
\end{Highlighting}
\end{Shaded}

\begin{verbatim}
##  [1] "Jugador"        "Y"              "X"              "Equipos_estado"
##  [5] "Posicion_1B"    "Posicion_2B"    "Posicion_3B"    "Posicion_C"    
##  [9] "Posicion_CF"    "Posicion_DH"    "Posicion_LF"    "Posicion_RF"   
## [13] "Posicion_RP"    "Posicion_RP/CL" "Posicion_SP"    "Posicion_SS"   
## [17] "Equipo_ARI"     "Equipo_ATL"     "Equipo_BAL"     "Equipo_BOS"    
## [21] "Equipo_CHC"     "Equipo_CHW"     "Equipo_CIN"     "Equipo_COL"    
## [25] "Equipo_DET"     "Equipo_HOU"     "Equipo_KC"      "Equipo_LAA"    
## [29] "Equipo_LAD"     "Equipo_MIA"     "Equipo_MIL"     "Equipo_MIN"    
## [33] "Equipo_NYM"     "Equipo_NYY"     "Equipo_OAK"     "Equipo_PHI"    
## [37] "Equipo_PIT"     "Equipo_SD"      "Equipo_SEA"     "Equipo_SF"     
## [41] "Equipo_STL"     "Equipo_TB"      "Equipo_TEX"     "Equipo_TOR"    
## [45] "Equipo_WSH"
\end{verbatim}

Cambiemos el nombre de la columna ``Posicion\_RP/CL'' por
``Posicion\_RP\_CL'' para evitar problemas en los algoritmos:

\begin{Shaded}
\begin{Highlighting}[]
\FunctionTok{names}\NormalTok{(dummy)[}\FunctionTok{names}\NormalTok{(dummy) }\SpecialCharTok{==} \StringTok{\textquotesingle{}Posicion\_RP/CL\textquotesingle{}}\NormalTok{] }\OtherTok{\textless{}{-}} \StringTok{\textquotesingle{}Posicion\_RP\_CL\textquotesingle{}}
\end{Highlighting}
\end{Shaded}

\begin{Shaded}
\begin{Highlighting}[]
\NormalTok{formula\_iv }\OtherTok{\textless{}{-}}\NormalTok{ Y }\SpecialCharTok{\textasciitilde{}}\NormalTok{ X }\SpecialCharTok{|}\NormalTok{ Equipos\_estado }\SpecialCharTok{+}\NormalTok{ Posicion\_1B }\SpecialCharTok{+}\NormalTok{ Posicion\_2B }\SpecialCharTok{+}\NormalTok{ Posicion\_3B }\SpecialCharTok{+}\NormalTok{ Posicion\_C }\SpecialCharTok{+}\NormalTok{ Posicion\_CF }\SpecialCharTok{+}\NormalTok{ Posicion\_DH }\SpecialCharTok{+}\NormalTok{ Posicion\_LF }\SpecialCharTok{+}\NormalTok{ Posicion\_RF }\SpecialCharTok{+}\NormalTok{ Posicion\_RP }\SpecialCharTok{+}\NormalTok{ Posicion\_RP\_CL }\SpecialCharTok{+}\NormalTok{  Posicion\_SP }\SpecialCharTok{+}\NormalTok{ Posicion\_SS }\SpecialCharTok{+}\NormalTok{ Equipo\_ARI }\SpecialCharTok{+}\NormalTok{ Equipo\_ATL }\SpecialCharTok{+}\NormalTok{ Equipo\_BAL }\SpecialCharTok{+}\NormalTok{ Equipo\_BOS }\SpecialCharTok{+}\NormalTok{ Equipo\_CHC }\SpecialCharTok{+}\NormalTok{ Equipo\_CHW }\SpecialCharTok{+}\NormalTok{ Equipo\_CIN }\SpecialCharTok{+}\NormalTok{  Equipo\_COL }\SpecialCharTok{+}\NormalTok{ Equipo\_DET }\SpecialCharTok{+}\NormalTok{ Equipo\_HOU }\SpecialCharTok{+}\NormalTok{  Equipo\_KC  }\SpecialCharTok{+}\NormalTok{  Equipo\_LAA }\SpecialCharTok{+}\NormalTok{  Equipo\_LAD }\SpecialCharTok{+}\NormalTok{  Equipo\_MIA }\SpecialCharTok{+}\NormalTok{ Equipo\_MIL }\SpecialCharTok{+}\NormalTok{ Equipo\_MIN }\SpecialCharTok{+}\NormalTok{  Equipo\_NYM }\SpecialCharTok{+}\NormalTok{  Equipo\_NYY }\SpecialCharTok{+}\NormalTok{   Equipo\_OAK }\SpecialCharTok{+}\NormalTok{ Equipo\_PHI }\SpecialCharTok{+}\NormalTok{ Equipo\_PIT }\SpecialCharTok{+}\NormalTok{   Equipo\_SD  }\SpecialCharTok{+}\NormalTok{  Equipo\_SEA }\SpecialCharTok{+}\NormalTok{  Equipo\_SF  }\SpecialCharTok{+}\NormalTok{  Equipo\_STL }\SpecialCharTok{+}\NormalTok{ Equipo\_TB  }\SpecialCharTok{+}\NormalTok{  Equipo\_TEX }\SpecialCharTok{+}\NormalTok{ Equipo\_TOR }\SpecialCharTok{+}\NormalTok{ Equipo\_WSH }
\NormalTok{iv\_model }\OtherTok{\textless{}{-}} \FunctionTok{ivreg}\NormalTok{(formula\_iv, }\AttributeTok{data =}\NormalTok{ dummy)}
\FunctionTok{summary}\NormalTok{(iv\_model, }\AttributeTok{diagnostics=}\ConstantTok{TRUE}\NormalTok{)}
\end{Highlighting}
\end{Shaded}

\begin{verbatim}
## 
## Call:
## ivreg(formula = formula_iv, data = dummy)
## 
## Residuals:
##      Min       1Q   Median       3Q      Max 
## -3462.88  -562.46    89.31   593.83  2908.09 
## 
## Coefficients:
##             Estimate Std. Error t value Pr(>|t|)  
## (Intercept)  -158.47      68.33  -2.319   0.0209 *
## X             117.91     168.14   0.701   0.4836  
## 
## Diagnostic tests:
##                  df1 df2 statistic     p-value    
## Weak instruments  39 323     2.724 0.000000773 ***
## Wu-Hausman         1 360     0.433       0.511    
## Sargan            41  NA    44.823       0.315    
## ---
## Signif. codes:  0 '***' 0.001 '**' 0.01 '*' 0.05 '.' 0.1 ' ' 1
## 
## Residual standard error: 1015 on 361 degrees of freedom
## Multiple R-Squared: -0.003463,   Adjusted R-squared: -0.006243 
## Wald test: 0.4917 on 1 and 361 DF,  p-value: 0.4836
\end{verbatim}

En comparación con el modelo sin isntrumentos -donde se obtuvo un
\(p-value = 0.738\)-, aquí mejoró dicho valor. Sin embargo, no es
suficiente para que la variable \(X\) se estadísticamente significativa
para el modelo. También dio positivo para la prueba de instrumentos
débiles.

\end{document}
