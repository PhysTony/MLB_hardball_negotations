% Options for packages loaded elsewhere
\PassOptionsToPackage{unicode}{hyperref}
\PassOptionsToPackage{hyphens}{url}
%
\documentclass[
]{article}
\usepackage{amsmath,amssymb}
\usepackage{lmodern}
\usepackage{ifxetex,ifluatex}
\ifnum 0\ifxetex 1\fi\ifluatex 1\fi=0 % if pdftex
  \usepackage[T1]{fontenc}
  \usepackage[utf8]{inputenc}
  \usepackage{textcomp} % provide euro and other symbols
\else % if luatex or xetex
  \usepackage{unicode-math}
  \defaultfontfeatures{Scale=MatchLowercase}
  \defaultfontfeatures[\rmfamily]{Ligatures=TeX,Scale=1}
\fi
% Use upquote if available, for straight quotes in verbatim environments
\IfFileExists{upquote.sty}{\usepackage{upquote}}{}
\IfFileExists{microtype.sty}{% use microtype if available
  \usepackage[]{microtype}
  \UseMicrotypeSet[protrusion]{basicmath} % disable protrusion for tt fonts
}{}
\makeatletter
\@ifundefined{KOMAClassName}{% if non-KOMA class
  \IfFileExists{parskip.sty}{%
    \usepackage{parskip}
  }{% else
    \setlength{\parindent}{0pt}
    \setlength{\parskip}{6pt plus 2pt minus 1pt}}
}{% if KOMA class
  \KOMAoptions{parskip=half}}
\makeatother
\usepackage{xcolor}
\IfFileExists{xurl.sty}{\usepackage{xurl}}{} % add URL line breaks if available
\IfFileExists{bookmark.sty}{\usepackage{bookmark}}{\usepackage{hyperref}}
\hypersetup{
  pdftitle={Cambio estructural},
  pdfauthor={Antonio Huerta Montellano},
  hidelinks,
  pdfcreator={LaTeX via pandoc}}
\urlstyle{same} % disable monospaced font for URLs
\usepackage[margin=1in]{geometry}
\usepackage{color}
\usepackage{fancyvrb}
\newcommand{\VerbBar}{|}
\newcommand{\VERB}{\Verb[commandchars=\\\{\}]}
\DefineVerbatimEnvironment{Highlighting}{Verbatim}{commandchars=\\\{\}}
% Add ',fontsize=\small' for more characters per line
\usepackage{framed}
\definecolor{shadecolor}{RGB}{248,248,248}
\newenvironment{Shaded}{\begin{snugshade}}{\end{snugshade}}
\newcommand{\AlertTok}[1]{\textcolor[rgb]{0.94,0.16,0.16}{#1}}
\newcommand{\AnnotationTok}[1]{\textcolor[rgb]{0.56,0.35,0.01}{\textbf{\textit{#1}}}}
\newcommand{\AttributeTok}[1]{\textcolor[rgb]{0.77,0.63,0.00}{#1}}
\newcommand{\BaseNTok}[1]{\textcolor[rgb]{0.00,0.00,0.81}{#1}}
\newcommand{\BuiltInTok}[1]{#1}
\newcommand{\CharTok}[1]{\textcolor[rgb]{0.31,0.60,0.02}{#1}}
\newcommand{\CommentTok}[1]{\textcolor[rgb]{0.56,0.35,0.01}{\textit{#1}}}
\newcommand{\CommentVarTok}[1]{\textcolor[rgb]{0.56,0.35,0.01}{\textbf{\textit{#1}}}}
\newcommand{\ConstantTok}[1]{\textcolor[rgb]{0.00,0.00,0.00}{#1}}
\newcommand{\ControlFlowTok}[1]{\textcolor[rgb]{0.13,0.29,0.53}{\textbf{#1}}}
\newcommand{\DataTypeTok}[1]{\textcolor[rgb]{0.13,0.29,0.53}{#1}}
\newcommand{\DecValTok}[1]{\textcolor[rgb]{0.00,0.00,0.81}{#1}}
\newcommand{\DocumentationTok}[1]{\textcolor[rgb]{0.56,0.35,0.01}{\textbf{\textit{#1}}}}
\newcommand{\ErrorTok}[1]{\textcolor[rgb]{0.64,0.00,0.00}{\textbf{#1}}}
\newcommand{\ExtensionTok}[1]{#1}
\newcommand{\FloatTok}[1]{\textcolor[rgb]{0.00,0.00,0.81}{#1}}
\newcommand{\FunctionTok}[1]{\textcolor[rgb]{0.00,0.00,0.00}{#1}}
\newcommand{\ImportTok}[1]{#1}
\newcommand{\InformationTok}[1]{\textcolor[rgb]{0.56,0.35,0.01}{\textbf{\textit{#1}}}}
\newcommand{\KeywordTok}[1]{\textcolor[rgb]{0.13,0.29,0.53}{\textbf{#1}}}
\newcommand{\NormalTok}[1]{#1}
\newcommand{\OperatorTok}[1]{\textcolor[rgb]{0.81,0.36,0.00}{\textbf{#1}}}
\newcommand{\OtherTok}[1]{\textcolor[rgb]{0.56,0.35,0.01}{#1}}
\newcommand{\PreprocessorTok}[1]{\textcolor[rgb]{0.56,0.35,0.01}{\textit{#1}}}
\newcommand{\RegionMarkerTok}[1]{#1}
\newcommand{\SpecialCharTok}[1]{\textcolor[rgb]{0.00,0.00,0.00}{#1}}
\newcommand{\SpecialStringTok}[1]{\textcolor[rgb]{0.31,0.60,0.02}{#1}}
\newcommand{\StringTok}[1]{\textcolor[rgb]{0.31,0.60,0.02}{#1}}
\newcommand{\VariableTok}[1]{\textcolor[rgb]{0.00,0.00,0.00}{#1}}
\newcommand{\VerbatimStringTok}[1]{\textcolor[rgb]{0.31,0.60,0.02}{#1}}
\newcommand{\WarningTok}[1]{\textcolor[rgb]{0.56,0.35,0.01}{\textbf{\textit{#1}}}}
\usepackage{graphicx}
\makeatletter
\def\maxwidth{\ifdim\Gin@nat@width>\linewidth\linewidth\else\Gin@nat@width\fi}
\def\maxheight{\ifdim\Gin@nat@height>\textheight\textheight\else\Gin@nat@height\fi}
\makeatother
% Scale images if necessary, so that they will not overflow the page
% margins by default, and it is still possible to overwrite the defaults
% using explicit options in \includegraphics[width, height, ...]{}
\setkeys{Gin}{width=\maxwidth,height=\maxheight,keepaspectratio}
% Set default figure placement to htbp
\makeatletter
\def\fps@figure{htbp}
\makeatother
\setlength{\emergencystretch}{3em} % prevent overfull lines
\providecommand{\tightlist}{%
  \setlength{\itemsep}{0pt}\setlength{\parskip}{0pt}}
\setcounter{secnumdepth}{-\maxdimen} % remove section numbering
\ifluatex
  \usepackage{selnolig}  % disable illegal ligatures
\fi

\title{Cambio estructural}
\author{Antonio Huerta Montellano}
\date{15 de junio del 2022}

\begin{document}
\maketitle

Importando las librerías:

\section{Bases de datos}

Veamos cuál es el directorio de trabajo

\begin{Shaded}
\begin{Highlighting}[]
\FunctionTok{getwd}\NormalTok{()}
\end{Highlighting}
\end{Shaded}

\begin{verbatim}
## [1] "/home/usuario/Documentos/Github/Proyectos/MLB_HN/Models/Linear_models/Free_agent/Test"
\end{verbatim}

Cambiemos el directorio de trabajo y carguemos las bases de datos para
el modelo lineal en el mismo chunk:

\begin{Shaded}
\begin{Highlighting}[]
\FunctionTok{setwd}\NormalTok{(}\StringTok{"\textasciitilde{}/Documentos/Github/Proyectos/MLB\_HN/"}\NormalTok{)}
\NormalTok{free\_agents }\OtherTok{\textless{}{-}} \FunctionTok{read.csv}\NormalTok{(}\StringTok{\textquotesingle{}Data/New\_Data/Models/Article/article\_linear\_regression\_fa.csv\textquotesingle{}}\NormalTok{)}
\NormalTok{no\_free\_agents }\OtherTok{\textless{}{-}} \FunctionTok{read.csv}\NormalTok{(}\StringTok{\textquotesingle{}Data/New\_Data/Models/Article/article\_linear\_regression\_nfa.csv\textquotesingle{}}\NormalTok{)}
\end{Highlighting}
\end{Shaded}

Observemos el contenido de las bases de datos de los agentes libres:

\begin{Shaded}
\begin{Highlighting}[]
\FunctionTok{head}\NormalTok{(free\_agents)}
\end{Highlighting}
\end{Shaded}

\begin{verbatim}
##           Jugador          Y          X
## 1    A.J. Burnett  -471.0402  0.6333714
## 2 A.J. Pierzynski   199.1753 -0.5233840
## 3 A.J. Pierzynski  1218.5682 -0.6992366
## 4 A.J. Pierzynski -2515.7319  0.7312696
## 5    Aaron Harang   509.2721 -0.6291943
## 6    Adam LaRoche   293.4260  0.6825397
\end{verbatim}

Ahora de los que no son agentes libres:

\begin{Shaded}
\begin{Highlighting}[]
\FunctionTok{head}\NormalTok{(no\_free\_agents)}
\end{Highlighting}
\end{Shaded}

\begin{verbatim}
##        Jugador          Y          X
## 1  A.J. Achter   5.219730 -0.5472425
## 2  A.J. Achter  -4.778685  0.6529036
## 3 A.J. Burnett -31.897419  0.6282596
## 4    A.J. Cole  12.106822  0.7506655
## 5    A.J. Cole   4.863402  0.7651183
## 6    A.J. Cole  -4.778685  0.6628217
\end{verbatim}

\section{Test para cambio estructural}

Construyamos los modelos lineales correspondientes:

\begin{Shaded}
\begin{Highlighting}[]
\NormalTok{fa\_model }\OtherTok{\textless{}{-}} \FunctionTok{lm}\NormalTok{(Y }\SpecialCharTok{\textasciitilde{}}\NormalTok{ X, }\AttributeTok{data =}\NormalTok{ free\_agents)}
\FunctionTok{summary}\NormalTok{(fa\_model)}
\end{Highlighting}
\end{Shaded}

\begin{verbatim}
## 
## Call:
## lm(formula = Y ~ X, data = free_agents)
## 
## Residuals:
##     Min      1Q  Median      3Q     Max 
## -3546.5  -571.5   122.6   595.6  2818.0 
## 
## Coefficients:
##             Estimate Std. Error t value Pr(>|t|)  
## (Intercept)  -134.05      57.26  -2.341   0.0198 *
## X              21.98      83.50   0.263   0.7925  
## ---
## Signif. codes:  0 '***' 0.001 '**' 0.01 '*' 0.05 '.' 0.1 ' ' 1
## 
## Residual standard error: 1013 on 361 degrees of freedom
## Multiple R-squared:  0.0001919,  Adjusted R-squared:  -0.002578 
## F-statistic: 0.06928 on 1 and 361 DF,  p-value: 0.7925
\end{verbatim}

\begin{Shaded}
\begin{Highlighting}[]
\NormalTok{no\_fa\_model }\OtherTok{\textless{}{-}} \FunctionTok{lm}\NormalTok{(Y }\SpecialCharTok{\textasciitilde{}}\NormalTok{ X, }\AttributeTok{data =}\NormalTok{ no\_free\_agents)}
\FunctionTok{summary}\NormalTok{(no\_fa\_model)}
\end{Highlighting}
\end{Shaded}

\begin{verbatim}
## 
## Call:
## lm(formula = Y ~ X, data = no_free_agents)
## 
## Residuals:
##     Min      1Q  Median      3Q     Max 
## -4113.9  -170.5  -116.2   136.1  3261.6 
## 
## Coefficients:
##             Estimate Std. Error t value             Pr(>|t|)    
## (Intercept)  146.136      6.303  23.186 < 0.0000000000000002 ***
## X             45.871      9.668   4.744           0.00000214 ***
## ---
## Signif. codes:  0 '***' 0.001 '**' 0.01 '*' 0.05 '.' 0.1 ' ' 1
## 
## Residual standard error: 508.8 on 6582 degrees of freedom
## Multiple R-squared:  0.003408,   Adjusted R-squared:  0.003257 
## F-statistic: 22.51 on 1 and 6582 DF,  p-value: 0.000002135
\end{verbatim}

Por último, determinemos si hay cambio estructural:

\begin{Shaded}
\begin{Highlighting}[]
\NormalTok{Y\_1 }\OtherTok{\textless{}{-}}\NormalTok{ free\_agents }\SpecialCharTok{\%\textgreater{}\%} \FunctionTok{select}\NormalTok{(Y)}
\NormalTok{X\_1 }\OtherTok{\textless{}{-}}\NormalTok{ free\_agents }\SpecialCharTok{\%\textgreater{}\%} \FunctionTok{select}\NormalTok{(X)}

\NormalTok{Y\_2 }\OtherTok{\textless{}{-}}\NormalTok{ no\_free\_agents }\SpecialCharTok{\%\textgreater{}\%} \FunctionTok{select}\NormalTok{(Y)}
\NormalTok{X\_2 }\OtherTok{\textless{}{-}}\NormalTok{ no\_free\_agents }\SpecialCharTok{\%\textgreater{}\%} \FunctionTok{select}\NormalTok{(X)}

\FunctionTok{chow.test}\NormalTok{(Y\_1, X\_1, Y\_2, X\_2)}
\end{Highlighting}
\end{Shaded}

\begin{verbatim}
##                          F value                            d.f.1 
##   47.036082097642676558280072641    2.000000000000000000000000000 
##                            d.f.2                          P value 
## 6943.000000000000000000000000000    0.000000000000000000005124364
\end{verbatim}

Como el p-value tiende a cero y por ende se rechaza la hipótesis nula
para un nivel de significancia de 0.001, entonces implica que hay un
cambio estructural entre estos modelos.

\end{document}
